%==============================================================================
\documentclass[cic,tc]{iiufrgs}

\usepackage[utf8]{inputenc}   % acentuação
\usepackage{graphicx}         % figuras
\usepackage{times}            % fonte Adobe Times
\usepackage[alf,abnt-emphasize=bf]{abntex2cite}	% cit abnt

%==============================================================================
\title{Reconhecendo Caracteres Escritos à Mão Utilizando Algorítimos de Deep Learning}
\author{Ficagna}{Alan}
\advisor[Prof.~Dr.]{Engel}{Paulo Martins}
\date{julho}{2015}
\keyword{IA}
\keyword{deeplearning}

%==============================================================================
\begin{document}
\maketitle

%==============================================================================
\begin{abstract}

  Este documento é uma revisão do estado da arte em relação as técnicas de
Deep Learning utilizadas no reconhecimento de caracteres escritos à mão, tendo o
banco de dados MNIST como meio de comparação. Além disso são apresentadas
algumas variações nos meta parâmetros do algoritmo tais como <TODO> para
avaliar o impacto no tempo de execução e na qualidade do resultado.

\end{abstract}

\begin{englishabstract}{Recognizing Handwritten Characters Using Deep Learning Algorithms}{IA, deeplearning} % tittle, keywords

  This document is a review of the state of the art in regards to Deep Learning
techniques used in handwritten characters recognition having the MNIST
database for comparison. Also, the impact in the change of parameters such as
<TODO> was measured to provide and indication of its effects on run speed and
quality of result.

\end{englishabstract}

%==============================================================================
\listoffigures
\listoftables
\begin{listofabbrv}{MNIST} % largest
 \item[MNIST] Mixed National Institute of Standards and Technology
 \item[CNN] Convolutional Neural Network
 \item[GPU] Graphical Processing Unit
\end{listofabbrv}
\tableofcontents

%==============================================================================
\chapter{Introdução}

Nos últimos anos houve um grande avanço em áreas como a Computação
Visual graças ao ressurgimento de um campo da Inteligência Artificial conhecido
como Redes Neurais. Este progresso foi responsável pelo despertar do interesse
e investimentos de grandes empresas como Google, Facebook e outros que apostam
no desenvolvimento de aplicações que há poucas décadas atrás eram apenas tema
de ficção científica tais como scanners faciais de alta precisão,
reconhecimento de objetos em cenas, carros que não precisam de motorista e
vários tipos de classificadores em geral. Essas novas técnicas ficaram
conhecidas como Deep Learning e prometem fazer uso da enorme quantidade de
dados e informações disponíveis na internet e nas bases de dados das empresas.

Este progresso foi causado por fatores como o incremento da capacidade de
processamento dos computadores e a ubiquidade de tecnologias como GPUs que
facilitaram o processamento vetorial e paralelo de imagens, sons e outros
domínios de alta dimensionalidade que até então possuíam um custo computacional
proibitivo. Além disso, novas contribuições teóricas também podem ser apontadas
como responsáveis pela melhoria de desempenho tais como aprendizado semi
supervisionado, uso de aprendizado baseado em energia <TODO COMPLETE>, <CITE>

\section{Objetivos}

O objetivo deste trabalho consiste em efetuar uma revisão do
histórico do desenvolvimento das técnicas de Deep Learning e um resumo do seu
estado da arte com a finalidade de adquirir proficiência no uso das mesmas.
Além disso, serão apresentados resultados práticos obtidos com a execução de
uma modelo que utiliza Deep Learning chamado de Rede Neural Convolucional
(CNN) no reconhecimento de caracteres escritos à mão utilizando o algoritmo da
rede LeNet de LeCun et al sobre a base MNIST.\@

\section{Revisão}

\subsection{Redes Neurais}

\subsubsection{Inteligência Artificial}

Discussões sobre a natureza da inteligência ou o que distingue um ser humano
dos outros animais ou mesmo de outros objetos inanimados é tema de constantes
debates desde pelo menos a época de Thomas Hobbes e sua filosofia mecanista que
acreditava que o ser humano era como uma máquina. Porém, não é objetivo deste
trabalho entrar nestas discussões e portanto o termo Inteligência Artificial
aqui é usado puramente em seu sentido computacional embora reconheça as enormes
implicações que descobertas nesta área podem vir a ter em outras disciplinas
como a Filosofia.

Como campo de pesquisa acadêmica, a Inteligência Artificial é geralmente
reconhecida como originada na conferência em Dartmouth que reuniu 10
pesquisadores e matemáticos que se tornariam líderes reconhecidos
mundialmente na área.\cite{russell2003norvig}


  Depth of architecture refers to the number of levels of composition of
  non-linear operations in the func- tion learned. Whereas most current
  learning algorithms correspond to shallow architectures (1, 2 or 3 levels),
  the mammal brain is organized in a deep architecture (Serre, Kreiman, Kouh,
  Cadieu, Knoblich, & Poggio, 2007) with a given input percept represented at
  multiple levels of abstraction, each level corresponding to a different area
  of cortex.  Humans often describe such concepts in hierarchical ways, with
  multiple levels of abstraction. The brain also appears to process information
  through multiple stages of transformation and representation. This is
  particularly clear in the primate visual system (Serre et al., 2007), with
  its sequence of processing stages: detection of edges, primitive shapes, and
  moving up to gradually more complex visual shapes.\cite{bengio2009learning}

  More precisely, functions that can be compactly represented by a depth k
  architecture might require an exponential number of computational elements to
  be represented by a depth k − 1 architecture. Since the number of
  computational elements one can afford depends on the number of training
  examples available to tune or select them, the consequences are not just
  computational but also statistical: poor generalization may be expected when
  using an insufficiently deep architecture for representing some
  functions.\cite{bengio2009learning}

  The most formal arguments about the power of deep architectures come from
  investigations into computa- tional complexity of circuits. The basic
  conclusion that these results suggest is that when a function can be
  compactly represented by a deep architecture, it might need a very large
  architecture to be represented by an insufficiently deep
  one.\cite{bengio2009learning}

  After having motivated the need for deep architectures that are non-local
  estimators, we now turn to the difficult problem of training them.
  Experimental evidence suggests that training deep architectures is more
  difficult than training shallow architectures (Bengio et al., 2007; Erhan,
  Manzagol, Bengio, Bengio, & Vin- cent, 2009).\cite{bengio2009learning}

  Until 2006, deep architectures have not been discussed much in the machine
  learning literature, because of poor training and generalization errors
  generally obtained (Bengio et al., 2007) using the standard random
  initialization of the parameters. Note that deep convolutional neural
  networks (LeCun, Boser, Denker, Hen- derson, Howard, Hubbard, & Jackel, 1989;
  Le Cun, Bottou, Bengio, & Haffner, 1998; Simard, Steinkraus, & Platt, 2003;
  Ranzato et al., 2007) were found easier to train, as discussed in Section
  4.5, for reasons that have yet to be really clarified.  Many unreported
  negative observations as well as the experimental results in Bengio et al.
  (2007), Erhan et al. (2009) suggest that gradient-based training of deep
  supervised multi-layer neural networks (starting from random initialization)
  gets stuck in “apparent local minima or plateaus”, and that as the
  architecture gets deeper, it becomes more difficult to obtain good
  generalization. When starting from random initializa- tion, the solutions
  obtained with deeper neural networks appear to correspond to poor solutions
  that perform worse than the solutions obtained for networks with 1 or 2
  hidden layers (Bengio et al., 2007; Larochelle, Bengio, Louradour, & Lamblin,
  2009).\cite{bengio2009learning}

  This happens even though k + 1-layer nets can easily represent what a k-layer
  net can represent (without much added capacity), whereas the converse is not
  true. How- ever, it was discovered (Hinton et al., 2006) that much better
  results could be achieved when pre-training each layer with an unsupervised
  learning algorithm, one layer after the other, starting with the first layer
  (that directly takes in input the observed x). The initial experiments used
  the RBM generative model for each layer (Hinton et al., 2006), and were
  followed by experiments yielding similar results using variations of
  auto-encoders for training each layer (Bengio et al., 2007; Ranzato et al.,
  2007; Vincent et al., 2008).  Most of these papers exploit the idea of greedy
  layer-wise unsupervised learning (developed in more de- tail in the next
  section): first train the lower layer with an unsupervised learning algorithm
  (such as one for the RBM or some auto-encoder), giving rise to an initial set
  of parameter values for the first layer of a neural network. Then use the
  output of the first layer (a new representation for the raw input) as input
  for another layer, and similarly initialize that layer with an unsupervised
  learning algorithm. After having thus initialized a number of layers, the
  whole neural network can be fine-tuned with respect to a supervised training
  criterion as usual.\cite{bengio2009learning}

\section{Figuras e tabelas}

%==============================================================================
\chapter{Estado da arte}

%==============================================================================
\chapter{Mais estado da arte}

%==============================================================================
\chapter{A minha contribuição}

%==============================================================================
\chapter{Prova de que a minha contribuição é válida}

%==============================================================================
\chapter{Conclusão}

%==============================================================================

\bibliographystyle{abntex2-alf}
\bibliography{biblio}

\end{document}

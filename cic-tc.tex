%==============================================================================
\documentclass[cic,tc]{iiufrgs}

\usepackage[utf8]{inputenc}   % acentuação
\usepackage{graphicx}         % figuras
\usepackage{times}            % fonte Adobe Times
\usepackage[alf,abnt-emphasize=bf]{abntex2cite}	% cit abnt

%==============================================================================
\title{Reconhecendo Caracteres Escritos à Mão Utilizando Algorítimos de DeepLearning}
\author{Ficagna}{Alan}
\advisor[Prof.~Dr.]{Engel}{Paulo Martins}
\date{julho}{2015}
\keyword{IA}
\keyword{deeplearning}

%==============================================================================
\begin{document}
\maketitle

%==============================================================================
\begin{abstract}

    Este documento é uma revisão do estado da arte em relação as técnicas de
    DeepLearning utilizadas no reconhecimento de caracteres escritos à mão,
    tendo o banco de dados MNIST como meio de comparação. Além disso são
    apresentadas algumas variações nos meta parâmetros do algoritmo tais como
    <TODO> para avaliar o impacto no tempo de execução e na qualidade do
    resultado.
\end{abstract}

\begin{englishabstract}{Recognizing Handwritten Characters Using DeepLearning Algorithms}{IA, deeplearning}

  This document is a review of the state of the art in regards to DeepLearning
  techniques used in handwritten characters reckognition having the MNIST
  database for comparsion. Also, the inpact in the change of paramaters such as
  <TODO> was measured to provide and indication of its effects on run speed and
  quality of result.

\end{englishabstract}

%==============================================================================
\listoffigures
\listoftables
\begin{listofabbrv}{MNIST}
 \item[MNIST] Mixed National Institute of Standards and Technology
\end{listofabbrv}
\tableofcontents

%==============================================================================
\chapter{Introdução}
\section{Figuras e tabelas}

%==============================================================================
\chapter{Estado da arte}

%==============================================================================
\chapter{Mais estado da arte}

%==============================================================================
\chapter{A minha contribuição}

%==============================================================================
\chapter{Prova de que a minha contribuição é válida}

%==============================================================================
\chapter{Conclusão}

%==============================================================================

\bibliographystyle{abntex2-alf}
\bibliography{biblio}

\end{document}

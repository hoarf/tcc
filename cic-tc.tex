%==============================================================================
\documentclass[cic,tc]{iiufrgs}

\usetocstyle{standard}
\usepackage[utf8]{inputenc}   % acentuação
\usepackage{lmodern}          % allows arbitrary font size
\usepackage{graphicx}         % figuras
\usepackage{times}            % fonte Adobe Times
\usepackage{hyperref}
\usepackage[alf,abnt-emphasize=bf]{abntex2cite}	% cit abnt
\usepackage{mathtools}	      % math symbols
\usepackage{amsthm}           % theorems and definitionsi
\usepackage{listings}         % codeblocks

% chktex latex syntax correction tool configurations
% chktex-file 18 disable warnings about "
% chktex-file 19 disable warnings about ô
% chktex-file 6 disable italic correction warnings

%==============================================================================
\title{Classificação de imagens utilizando deep supervised learning}
\author{Ficagna}{Alan}
\advisor[Prof.~Dr.]{Engel}{Paulo Martins}
\date{}{2015}
\keyword{IA}
\keyword{inteligência artificial}
\keyword{deeplearning}

%==============================================================================

\graphicspath{{fig/}}

\DeclareGraphicsExtensions{.pdf,.png,.jpg}

\begin{document}

\maketitle

\clearpage
\begin{flushright}
  \mbox{}\vfill
  {\sffamily\itshape{}
    ``Virtue is more to be feared than vice,\\
    because its excesses are not subject to the regulation of consciousness''\\}
  --- \textsc{Adam Smith}
\end{flushright}

%==============================================================================
\begin{abstract}

Este documento é uma revisão do estado da arte em relação as técnicas de
Deep Learning utilizadas no reconhecimento de caracteres escritos à mão, tendo o
banco de dados MNIST como meio de comparação. Além disso são apresentadas
algumas variações nos meta parâmetros do algoritmo tais como <TODO> para
avaliar o impacto no tempo de execução e na qualidade do resultado.
\end{abstract}

\begin{englishabstract}{Recognizing Handwritten Characters Using Deep Learning Algorithms}{IA, deeplearning} % tittle, keywords

This document is a review of the state of the art in regards to Deep Learning
techniques used in handwritten characters recognition having the MNIST
database for comparison. Also, the impact in the change of parameters such as
<TODO> was measured to provide and indication of its effects on run speed and
quality of result.
\end{englishabstract}

%==============================================================================
\listoffigures
\listoftables
\begin{listofabbrv}{ILSVRC} % largest
 \item[MNIST] Mixed National Institute of Standards and Technology
 \item[CNN] Convolutional Neural Network
 \item[GPU] Graphical Processing Unit
 \item[ReLU] Rectified Linear Unit
 \item[CONV] Convolutional
 \item[FC] Fully Connected
 \item[ILSVRC] Large Scale Visual Recognition Challenge
\end{listofabbrv}
\tableofcontents

%==============================================================================

\chapter{Introdução}

Nos últimos anos houve um grande avanço na Computação Visual graças ao
ressurgimento de um campo da Inteligência Artificial conhecido como Redes
Neurais. Este progresso foi responsável pelo despertar do interesse e
investimentos de grandes empresas como Google, Facebook e outros que apostam no
desenvolvimento de aplicações que há poucas décadas atrás eram apenas tema de
ficção científica tais como scanners faciais de alta precisão, reconhecimento
de objetos em cenas, carros que não precisam de motorista e vários tipos de
classificadores em geral. Essas novas técnicas ficaram conhecidas como Deep
Learning e prometem fazer uso da enorme quantidade de dados e informações
disponíveis na internet e nas bases de dados das empresas.

Este progresso foi causado por fatores como o incremento da capacidade de
processamento dos computadores e a ubiquidade de tecnologias como GPUs que
facilitaram o processamento vetorial e paralelo de imagens, sons e outros
domínios de alta dimensionalidade que até então possuíam um custo computacional
proibitivo.

\section{Objetivos}

<REVISAR>
O objetivo deste trabalho consiste em efetuar uma revisão do
histórico do desenvolvimento das técnicas de Deep Learning e um resumo do seu
estado da arte com a finalidade de adquirir proficiência no uso das mesmas.
Além disso, serão apresentados resultados práticos obtidos com a execução de
uma modelo que utiliza Deep Learning chamado de Rede Neural Convolucional
(CNN) no reconhecimento de caracteres escritos à mão utilizando o algoritmo da
rede LeNet de LeCun et al sobre a base MNIST.\@



% chktex-file 18 disable warnings about "
% chktex-file 19 disable warnings about ô
% chktex-file 6 disable italic correction warnings

\chapter{Histórico}

\section{Inteligência Artificial}

Definir o que é inteligência não é uma tarefa fácil, e inteligência artificial
é mais difícil ainda. Historicamente, existiram duas abordagens para a
definição de inteligência: O comparativo com humanos e o comparativo com seres
racionais.

\subsection{Inteligência Humana}

No primeiro modelo, o seres da espécie humana são tidos como parâmetros e
portanto, qualquer avaliação de inteligência tem que se refletir em
comportamentos ou pensamentos similares aos humanos. O Teste de Turing proposto
pelo matemático Alan Turing (1950) foi uma maneira operacional de estabelecer
inteligência pelo comparativo comportamental. Em sua versão mais simplificada,
um computador poderia ser considerado inteligente se, ao ser interrogado por um
humano com algumas questões, este não pudesse determinar se as respostas eram
oriundas de um computador ou de uma pessoa. A Ciência Cognitiva, por outro
lado, busca compreender o mecanismo pelos quais os humanos são capazes de
pensar e à partir disso estabelecer comparações com algum algoritmo que
eventualmente pudesse replicar estas capacidades.

\subsection{Inteligência Racional}

Além do paradigma humano, existe um outro que é a comparação com seres
racionais. Um ser é dito racional se ele age da ``maneira correta'', dado o que
ele conhece. Isso nos leva a seguinte definição: um ser é inteligente se ele
age ou pensa como um ser racional.

Aristóteles foi o primeiro a sistematizar uma maneira de ``pensar
corretamente''.  Ele foi o criador da disciplina que hoje conhecemos por Lógica
que estabelecia estruturas argumentativas tais como: ``Sócrates é um homem;
todos os homens são mortais; portanto, Sócrates é mortal'' as quais sempre
produzem conclusões corretas. Porém, estes silogismas nem sempre são capazes de
traduzir conhecimento informal, principalmente em face à incertezas e por isso
tendem a não ser reconhecidos como tudo que existe em termos de inteligência.

Além disso existem diferenças entre resolver um problema em princípio e
resolvê-lo na prática. Embora a abordagem pelas ``leis do pensamento'' do
Aristóteles foquem em realizar inferências corretas, esta não é capaz de
exaurir o que significa agir como um ser racional pois existem situações em que
não há uma ação comprovadamente correta, mas em que uma ação deve ser tomada
mesmo assim. Além disso, existem comportamentos racionais que não podem ser
deduzidos por mera inferência como, por exemplo, o reflexo de remover
rapidamente a mão do fogo sem uma cuidadosa pré deliberação sobre o assunto.

O propósito deste trabalho não é exaurir todos os ângulos pelos quais esta
questão já foi abordada, tais como psicologia, lógica e filosofia mas sim
apresentar apenas uma introdução que permita entender o surgimento do o campo da
Inteligência Artificial dentro da ciência da computação.

\section{Histórico da Inteligência Artificial}

\subsection{Primeiros anos}

O primeiro trabalho a ser considerado efetivamente ``Inteligência Artificial''
foi feito em 1943 por Warren McCulloch e Walter Pits, no qual foi proposto o
modelo de funcionamento de um neurônio artificial.  Este neurônio podia assumir
os estados ``ligado'' e ``desligado'', determinado em resposta aos estímulos
proporcionados pelos outros neurônios vizinhos. Foi demonstrado também que uma
rede de neurônios artificial conseguia computar qualquer função computável e
sugerido que a mesma poderia aprender.

Em 1949 Donald Hebb demonstrou uma simples regra pela qual a força ou os pesos
numéricos de conexão entre os neurônios poderia ser modificada, a qual ficou
conhecida como aprendizado hebbiano. Logo em seguida, em 1950, foi construído o
primeiro computador de rede neural chamado de SNARC que fora feito com 3000
tubos de vácuo e conseguia simular uma rede de 40 neurônios.

Embora o neurônio artificial seja considerado o primeiro trabalho, o termo
Inteligência Artificial só viria a ser cunhado mais de 10 anos depois em
Darthmouth nos Estados Unidos em um workshop organizado por John McCarthy que
duraria dois meses e cujo propósito era estudar teoria dos autômatos, redes
neurais e inteligência. Embora não tenha produzido nenhum progresso, o workshop
serviu para introduzir as principais figuras da área entre si.

Nos seus primeiros anos a IA obteve bastante sucesso. Um a um, os items da
lista ``uma máquina nunca vai poder fazer X'' compilada por Turing iam sendo
desbancados. Newel and Simon criaram o primeiro GPS (General Problem Solver),
um programa desenhado para imitar a maneira como humanos pensam ao invés de
simplesmente seguir regras lógicas. Na classe limitada de desafios que
conseguia resolver, a ordem dos subobjetivos que o programa tentava seguir era
similar à humana. Na IBM, Nathaniel Rochester criaria o primeiro Provador de
Teoremas Geométricos, o qual conseguia provar teoremas que vários estudantes
consideravam complexos.  Arthur Samuel refutou a ideia de que programas somente
conseguiriam fazer o que lhes eram dito criando um programa de computador que
conseguia jogar um jogo melhor que o seu criador.

Foi nessa época também que McCarthy criaria sua linguagem de programação
\emph{Lisp} que viraria a linguagem dominante em IA e com a qual ele faria o
primeiro algoritmo que aceitava novos axiomas, sendo então o primeiro autômato
que conseguia adquirir novas capacidades sem necessitar ser reprogramado.
Outras contribuições significativas foram o programa SAINT (1963) que conseguia
resolver problemas de Cálculo típicos do primeiro ano de curso, o programa
ANALOGY (1968) respondia a perguntas de analogia geométrica como as de testes
de QI, o programa STUDENT (1967) resolvia problemas algébricos e o programa
SHRDLU (1972) era capaz de executar instruções do tipo ``Encontre um bloco
maior do que você está segurando agora e o coloque na caixa''. Foi nessa época
também que foram feitas melhorias no processo de aprendizado Hebbiano que
ficaram conhecidas como redes \emph{adalines}.

\subsection{Maturação e Estado da Arte}

O sucesso inicial foi seguido de um período de relativa estagnação. O fato de
que estes algoritmos só conseguiam fazer manipulação simbólica e não conheciam
nada sobre o domínio que trabalhavam provou ser um grande empecilho. Além disso
a maioria dos primeiros programas de IA resolviam o problema tentando várias
combinações até que o resultado fosse atingido, o que não era factível para
instâncias de problemas marginalmente maiores. Foi nesta época que Minsky e
Papert provaram que um neurônio artificial agora conhecido como
\emph{perceptron} era restrito no tipo de funções que conseguia representar e o
financiamento de pesquisas na area se tornou escasso. Já na década de 1970,
houve o surgimento dos sistemas baseados em conhecimento prévio como o programa
DENDRAL que viu uma melhora significativa de desempenho na tarefa de inferir a
estrutura molecular à partir da massa dos seus componentes. O programa que
inicialmente tentava todas as alternativas possíveis passou a utilizar de
conhecimento de especialistas em química para identificar subestruturas
conhecidas o que reduzia significadamente o número de tentativas.

O primeiro caso de sucesso comercial reportado destes sistemas especialistas
era um programa que ajudava na compra de novos computadores e que teria
economizado à empresa US\$40 milhões por ano, o que explica o fato destes
sistemas continuarem sendo usados até hoje.

Paralelamente, houve o ressurgimento das redes neurais no final da década de
1980, proporcionado pela reinvenção do algoritmo de \emph{back-propagation},
uma maneira eficiente de atualizar os pesos das conexões nos neurônios, e
também uma melhoria nas práticas de pesquisa na área que passou a exigir um
maior rigor científico acompanhado do uso de bancos de dados padrões que
permitiam o teste e comparação dos resultados entre diferentes algoritmos. Em
uma outra linha recente, a ênfase no algoritmo da lugar ao uso de grandes
quantidades de dados inspirado no trabalho de Yarowsky (1995) que demostrou ser
possível reconhecer se o uso da palavra \emph{plant} significava uma flor ou uma
fábrica com precisão acima de 96\% sem utilizar de rótulos fornecidos por
humanos, apenas com as definições do dicionário desta palavra. Trabalhos como o
dele sugerem um ``gargalo de conhecimento'' na IA, ou seja, o conhecimento que
o um sistema precisa pra resolver um problema pode vir de grandes massas de
dados ao invés de intervenções de especialistas na área. Hoje em dia, a IA está
em várias aplicações de campos diferentes, como por exemplo:

\begin{itemize}
\itemsep1pt\parskip0pt\parsep0pt
\item
  Na área de robótica veicular com carros auto guiados.
\item
  Na área de reconhecimento de fala.
\item
  Planejamento automático de tarefas.
\item
  Jogos, com destaque para o computador DEEP BLUE da IBM que derrotou o
  campeão mundial de xadrez, Garry Kasparov.
\item
  Controle de SPAN em mensagens de e-mail.
\item
  Militar e Logística.
\end{itemize}

\section{Redes Neurais}

As primeiras definições de uma rede neural artificial foram feitas em 1943 por
McCulloch e Pits inspiradas na hipótese de que as atividades mentais do cérebro
humano consistem em reações eletroquímicas em redes de células cerebrais
chamadas de neurônios. O neurônio artificial consiste em uma unidade
computacional abstrata capaz de receber estímulos de entrada e produzir uma ou
mais saídas e uma rede neural artificial --- de agora em diante referenciada
apenas por \emph{rede neural} --- consiste em um grafo direcionado cujos nodos
são neurônios artificiais.

\begin{figure}
  \caption{Abstração de um neurônio artificial}
  \begin{center}
    \includegraphics[scale=0.5]{placeholder}
  \end{center}
\end{figure}

\subsection{O neurônio artificial}

Cada neurônio recebe $m$ entradas ponderadas por um peso $w_m$,
correspondente a entrada do seu i-ésimo vizinho. Esta configuração foi
inspirada nos neurônios naturais cujas interconexões são reguladas por um
potencial de ativação responsável por diferenciar a intensidade entre as
mesmas. A saída é produzida aplicando-se a função limite $\sigma$ na soma
de todos as entradas ponderadas do neurônio. Esta função simula a \emph{lógica
  de limite} existente nos neurônios naturais, os quais somente irão propagar o
seu sinal caso haja um acúmulo suficiente de potencial elétrico no neurônio. Em
sua forma matemática, a saída $y$ do neurônio artificial pode ser expressada da
seguinte forma:

$$y=\sigma(\sum_{i=1}^{m}x_i w_i + b)$$

A função $\sigma$ representa na verdade uma classe de funções que podem ser
utilizadas dependendo do objetivo da construção da rede neural. Para problemas
de classificação \emph{linearmente separáveis}, pode ser usada a função limite
exemplificada na figura~\ref{fig:lim-functin}, enquanto para problemas não
linearmente separáveis é necessário utilizar algum tipo de função
\emph{sigmoide}. A função logística é um exemplo de função sigmoide
frequentemente utilizada por ser diferenciável em todo o seu domínio, o que
permite produzir garantias de convergência nos algoritmos de aprendizagem.

O valor $b$ corresponde ao \emph{bias} aplicado a função e pode ser omitido
simplesmente inserindo-se artificialmente uma nova entrada $x_0 = 1$.

\begin{figure}
\label{fig:lim-fun}
  \caption{Exemplo de funções de limite $\sigma$ usadas em neurônios artificiais}
  \begin{center}
    \includegraphics[scale=0.5]{placeholder}
  \end{center}
\end{figure}

\subsubsection{Bias}

Um neurônio artificial pode ser visto como um aproximador de funções. Mais
adiante<TODO, ref>, será explicado que tipo de função pode ser aproximada e em
qual situação, mas antes cabe aqui explicar o objetivo da entrada de
\emph{bias} ou \emph{viés}. Todo neurônio artificial é dito ter pelo menos uma
entrada extra $x_0 = 1$ e consequentemente, o peso extra associado $w_0$
chamado de bias $b$. O bias permite que a função aproximada pelo neurônio não
necessite obrigatoriamente passar pela origem $(0,0)$, o que reduz o erro médio
da aproximação.

\begin{figure}
  \caption{Exemplo de função aproximada com e sem o vetor de bias}
  \begin{center}
    \includegraphics[scale=0.5]{placeholder}
  \end{center}
\end{figure}

\subsubsection{Redes neurais}

Em uma rede neural, os neurônios são organizados em \emph{camadas}. Alguns
autores não consideram a camada de entrada como uma camada distinta, mas esta
abstração permite simplificar a notação e torna o processo de conectar redes
neurais entre si mais intuitivo. Outra camada especial é a de saída, pois é
ela que determina se a rede exercerá a função de um \emph{classificador} ou de
um \emph{regressor}. Todas as demais camadas intermediárias são chamadas de
\emph{camadas ocultas}; o nome não possui nenhuma conotação especial e serve
apenas para diferenciá-las das camadas de entrada e saída. Em uma rede
\emph{feed forward} as saídas de cada camada somente são computadas na direção
e sentido da camada de saída, isto é, não há uma retroalimentação das conexões
de saída para as de entrada. Quando isto acontece a rede é chamada de
\emph{recorrente} porém, o estudo deste tipo de rede neural não esta nos
objetivos deste trabalho.

Sendo assim, uma rede neural necessita ter pelo menos duas camadas --- a de
entrada e a de saída. A adição de camadas ou neurônios não possui um efeito
óbvio e generalizável para qualquer rede, e portanto é alvo de constante
experimentação e observação empírica sujeito a comportamento variável
dependendo do problema.

\begin{figure}
\label{fig:003-nn}
  \caption{Exemplo de uma rede neural de 3 camadas}
  \begin{center}
    \includegraphics[scale=0.5]{placeholder}
  \end{center}
\end{figure}

Cada camada $S$ é formada por $R$ neurônios, cujos pesos das conexões podem ser
representados através de uma matriz $W$. A vetorização da computação dos pesos
das camadas permite que operações mais eficientes sejam utilizadas para
realizar a computação da saída da rede neural e do ajuste dos pesos.

$$ W_{S,R} =
\begin{bmatrix}
  w_{1,1} & w_{1,2} & \cdots & w_{1,R} \\
  w_{2,1} & w_{2,2} & \cdots & w_{2,R} \\
  \vdots  & \vdots  & \ddots & \vdots  \\
  w_{S,1} & w_{S,2} & \cdots & w_{S,R}
\end{bmatrix}
$$

Com isto o resultado final da rede neural pode ser representado pela equação:

$$ a^{S} = \sigma(Wa^{S-1}+b)$$

O símbolo $a$ representa o vetor de \emph{ativação} dos neurônios da camada S.

\subsubsection{Aprendizagem}

A aprendizagem em uma rede neural se da pelo ajuste dos pesos $w$ das conexões
dos neurônios. Ao longo dos anos foram propostas várias maneiras de atualizar
os pesos de uma rede neural automaticamente.

\begin{figure}
\label{fig:002-limit-functions}
  \caption{Exemplo de funções de ativação}
  \begin{center}
    \includegraphics[height=8cm]{placeholder}
  \end{center}
  \legend{Na esquerda uma simples função de limite; à direita a função $\frac{1}{1+e^{-x}}$}
\end{figure}

Quando a função de ativação $g$ representada na~\ref{fig:001-ann} for uma
função limite[\ref{fig:002-limit-functions}], o neurônio artificial é chamado
de \emph{perceptron}; Quando a função de ativação $g$ for uma função logística,
o neurônio artificial é chamado de \emph{perceptron sigmoide} --- devido ao
fato de que a função logística é um tipo particular de sigmoide. Estas funções
são usadas para garantir que o neurônio consiga reproduzir funções não lineares
sendo a principal vantagem da função sigmoide a sua diferenciabilidade em todo
o seu domínio, fator que mais tarde será relevante na atualização dos pesos das
conexões na rede.


% chktex-file 18 disable warnings about "
% chktex-file 19 disable warnings about ô
% chktex-file 6 disable italic correction warnings


\chapter{Revisão sobre Redes Neurais}

As primeiras definições de uma rede neural artificial foram feitas em 1943 por
McCulloch e Pits inspiradas na hipótese de que as atividades mentais do cérebro
humano consistem em reações eletroquímicas em redes de células cerebrais
chamadas de neurônios. O neurônio artificial consiste em uma unidade
computacional abstrata capaz de receber estímulos de entrada e produzir uma ou
mais saídas e uma rede neural artificial --- de agora em diante referenciada
apenas por \emph{rede neural} --- consiste em um grafo direcionado cujos nodos
são neurônios artificiais.

\begin{figure}
  \caption{Abstração de um neurônio artificial}
  \begin{center}
    \includegraphics[scale=0.5]{placeholder}
  \end{center}
\end{figure}

\subsection{O neurônio artificial}

Cada neurônio recebe $m$ entradas ponderadas por um peso $w$,
correspondente a entrada do seu i-ésimo vizinho. Esta configuração foi
inspirada nos neurônios naturais cujas interconexões são reguladas por um
potencial de ativação responsável por diferenciar a intensidade entre as
mesmas. A saída é produzida aplicando-se a função limite $\sigma$ na soma
de todos as entradas ponderadas do neurônio. Esta função simula a \emph{lógica
  de limite} existente nos neurônios naturais, os quais somente irão propagar o
seu sinal caso haja um acúmulo suficiente de potencial elétrico no neurônio. Em
sua forma matemática, a saída $y$ do neurônio artificial pode ser expressada da
seguinte forma:

$$y=\sigma(\sum_{i=1}^{m}x_i w_i + b)$$

A função $\sigma$ representa na verdade uma classe de funções que podem ser
utilizadas dependendo do objetivo da construção da rede neural. Para problemas
de classificação \emph{linearmente separáveis}, pode ser usada a função limite
exemplificada na figura~\ref{fig:lim-functin}, enquanto para problemas não
linearmente separáveis é necessário utilizar algum tipo de função
\emph{sigmoide}. A função logística é um exemplo de função sigmoide
frequentemente utilizada por ser diferenciável em todo o seu domínio, o que
permite produzir garantias de convergência nos algoritmos de aprendizagem.

O valor $b$ corresponde ao \emph{bias} aplicado a função e pode ser omitido
simplesmente inserindo-se artificialmente uma nova entrada $x_0 = 1$.

\begin{figure}
\label{fig:lim-fun}
  \caption{Exemplo de funções de limite $\sigma$ usadas em neurônios artificiais}
  \begin{center}
    \includegraphics[scale=0.5]{placeholder}
  \end{center}
\end{figure}

\subsubsection{Bias}

Um neurônio artificial pode ser visto como um aproximador de funções. Mais
adiante<TODO, ref>, será explicado que tipo de função pode ser aproximada e em
qual situação, mas antes cabe aqui explicar o objetivo da entrada de
\emph{bias} ou \emph{viés}. Todo neurônio artificial é dito ter pelo menos uma
entrada extra $x_0 = 1$ e consequentemente, o peso extra associado $w_0$
chamado de bias $b$. O bias permite que a função aproximada pelo neurônio não
necessite obrigatoriamente passar pela origem $(0,0)$, o que reduz o erro médio
da aproximação caso os dados de treinamento não tenham sofrido nenhum tipo de
ajuste numérico como por exemplo a remoção do valor médio.

\begin{figure}
  \caption{Exemplo de função aproximada com e sem o vetor de bias}
  \begin{center}
    \includegraphics[scale=0.5]{placeholder}
  \end{center}
\end{figure}

\subsubsection{Redes neurais}

Em uma rede neural, os neurônios são organizados em \emph{camadas}. Alguns
autores não consideram a camada de entrada como uma camada distinta, mas esta
abstração permite simplificar a notação e torna o processo de conectar redes
neurais entre si mais intuitivo. Outra camada especial é a de saída, pois é
ela que determina se a rede exercerá a função de um \emph{classificador} ou de
um \emph{regressor}. Todas as demais camadas intermediárias são chamadas de
\emph{camadas ocultas}; o nome não possui nenhuma conotação especial e serve
apenas para diferenciá-las das camadas de entrada e saída. Em uma rede
\emph{feed forward} as saídas de cada camada somente são computadas na direção
e sentido da camada de saída, isto é, não há uma retroalimentação das conexões
de saída para as de entrada. Quando isto acontece a rede é chamada de
\emph{recorrente} porém, o estudo deste tipo de rede neural não esta nos
objetivos deste trabalho.

Sendo assim, uma rede neural necessita ter pelo menos duas camadas --- a de
entrada e a de saída. A adição de camadas ou neurônios não possui um efeito
óbvio e generalizável para qualquer rede, e portanto é alvo de constante
experimentação e observação empírica sujeito a comportamento variável
dependendo do problema.

\begin{figure}\label{fig:003-nn}
  \caption{Exemplo de uma rede neural de 3 camadas}
  \begin{center}
    \includegraphics[scale=0.5]{placeholder}
  \end{center}
\end{figure}

Cada camada $S$ é formada por $R$ neurônios, cujos pesos das conexões podem ser
representados através de uma matriz $W$. A notação em forma vetorial e
matricial dos pesos das camadas permite que operações mais eficientes sejam
utilizadas para realizar a computação da saída da rede neural e do ajuste dos
mesmos.

$$ W_{S,R} =
\begin{bmatrix}
  w_{1,1} & w_{1,2} & \cdots & w_{1,R} \\
  w_{2,1} & w_{2,2} & \cdots & w_{2,R} \\
  \vdots  & \vdots  & \ddots & \vdots  \\
  w_{S,1} & w_{S,2} & \cdots & w_{S,R}
\end{bmatrix}
$$

A saída ou \emph{ativação} de uma camada $a'$ pode ser expressa em função da
ativação da camada anterior $a$ da seguinte maneira:

$$a^{'} = \sigma(Wa+b)$$

\subsubsection{Uma intuição sobre aprendizagem}

A aprendizagem em uma rede neural se da pelo ajuste dos pesos $w$ das conexões
dos neurônios. Ao longo dos anos foram propostas várias maneiras de atualizar
os pesos de uma rede neural automaticamente através de uma \emph{regra de
  aprendizagem}. Estas regras foram categorizadas em três grandes grupos:

\begin{description}
  \itemsep1pt\parskip0pt\parsep0pt

  \item[Aprendizagem Supervisionada] As entradas são acompanhadas por um
    conjunto de valores corretamente associados a mesma.

  \item[Aprendizagem Por Reforço] Semelhante ao aprendizado supervisionado,
    porém ao invés da resposta correta, é fornecido algo como uma nota ou uma
    indicação da performance da predição.

  \item[Aprendizagem Não Supervisionada] Nenhuma informação adicional a
    respeito das entradas é fornecida ao algoritmo.

\end{description}

Considere o problema de classificar corretamente os exemplos da
figura~\ref{fig:percetron-problem-layout}. Os pontos representados no plano dos
pesos das conexões são:

$$ \mathbf{x_0} =
\begin{bmatrix}
  1\\
  2
\end{bmatrix}
\mathbf{x_1} = \begin{bmatrix}
  3\\
  4
\end{bmatrix}
\mathbf{x_2} = \begin{bmatrix}
  5\\
  6
\end{bmatrix}
$$
\begin{figure}\label{fig:perceptron-problem-layout}
  \caption{Figura com o espaço de solução}
  \begin{center}
    \includegraphics[height=8cm]{placeholder}
  \end{center}
\end{figure}

O primeiro passo na formulação de um algoritmo de aprendizagem é a estipulação
de um valor inicial para os pesos das conexões dos neurônios, que pode ser
aleatória ou baseada em algum valor pré conhecido a respeito do problema e
depende das propriedades numéricas do mesmo. Para o exemplo do
\emph{perceptron} acima considere o vetor de pesos $w_0^T=[0.8,0.3]$.

\begin{figure}\label{fig:simple-perceptron}
  \caption{Uma rede neural simples de um único neurônio}
  \begin{center}
    \includegraphics[height=8cm]{placeholder}
  \end{center}
\end{figure}

É necessário também escolher uma \emph{função limite} $\sigma$. Para o exemplo
acima foi escolhida a função $hardlim$:
i
$$
hardlim(n) = \begin{cases}
  0 & \quad \text{if~} n<0\\
  1 & \quad \text{if~} n>=0\\
  \end{cases}
$$

\begin{figure}\label{fig:limit-functions-exapmple}
  \caption{Exemplo de funções de ativação}
  \begin{center}
    \includegraphics[height=8cm]{placeholder}
  \end{center}
  \legend{Na esquerda uma simples função de limite; à direita a função $\frac{1}{1+e^{-x}}$}
\end{figure}

% Quando a função de ativação $g$ representada na~\ref{fig:001-ann} for uma
% função limite[\ref{fig:002-limit-functions}], o neurônio artificial é chamado
% de \emph{perceptron}; Quando a função de ativação $g$ for uma função logística,
% o neurônio artificial é chamado de \emph{perceptron sigmoide} --- devido ao
% fato de que a função logística é um tipo particular de sigmoide. Estas funções
% são usadas para garantir que o neurônio consiga reproduzir funções não lineares
% sendo a principal vantagem da função sigmoide a sua diferenciabilidade em todo
% o seu domínio, fator que mais tarde será relevante na atualização dos pesos das
% conexões na rede.


% chktex-file 18 disable warnings about "
% chktex-file 19 disable warnings about ô
% chktex-file 6 disable italic correction warnings

\section{Deep Learning}

\subsection{Por que usar deep learning}

Hoje sabemos\citet{hubel1959receptive} que o processamento visual no cérebro
humano ocorre de maneira gradual e que diferentes regiões são responsáveis pelo
processamento de diferentes \emph{features} extraídas do estímulo sensorial.
Primeiro são detectadas abstrações simples como bordas, que por sua vez são
combinadas em funções mais complexas como detecção de movimento e
reconhecimento facial. Embora esta seja uma boa fonte de inspiração para o uso
de \emph{deep neural networks}, pesquisadores da área hesitam em comparar o que
acontece em uma rede neural artificial com o que acontece dentro de um cérebro
natural, visto que o completo funcionamento do mesmo é um problema científico
em aberto.

A motivação para o uso de uma rede neural \emph{deep} ao invés das arquiteturas
tradicionais \emph{shallow}, tais como \emph{support vector machines} ou redes
neurais com poucas camadas, visto que estas conseguem simular qualquer outra, é
que existe um \emph{tradeoff} entre espaço de memória e tempo de computação que
pode ser explorado para acelerar o processamento das redes.  Uma maneira de
entender este \emph{tradeoff} em redes neurais, é imaginar a saída das camadas
escondidas como resultados intermediários da computação.
\emph{shallow}\cite{bengio2007scaling}

Além disso, até bem pouco tempo atras, os melhores resultados nas tarefas de
classificação e detecção de imagens, videos, sons e linguagem natural
necessitavam de um amplo conhecimento \emph{a priori} por parte dos cientistas
cujo trabalho era elaborar \emph{filtros} capazes de sintetizar toda a
complexidade destas tarefas.

Devido ao alto custo do desenvolvimento e pesquisa destes filtros, surge a
necessidade de um método que consiga aprendê-los de maneira automática, mas que
ainda assim mantenha a precisão dos filtros artesanais. Estes métodos hoje são
conhecidos como \emph{deep learning} e envolvem um gama variada de técnicas
para contornar os problemas de arquiteturas com múltiplas camadas, a fim de
extrair o benefício do tradeoff entre memória e computação mencionado.

\subsection{Problemas com deep learning}

Já que redes neurais já são conhecidas há algum tempo, é de se estranhar que
ninguém tenha tido a ideia de acrescentar mais camadas nestas redes
anteriormente. O fato é que simplesmente adicionar mais camadas incorre em
problemas que até alguns anos atrás eram difíceis de serem contornados.

\begin{description}

  \item[overfitting:] Uma rede com múltiplas camadas tende a apresentar um
    número maior de parâmetros que podem ser treinados, o que leva a um
    problema de \emph{overfitting}, ou seja, a rede simplesmente aprende o
    conjunto de dados de treinamento, mas não é muito capaz de realizar boas
    inferências sobre novos conjuntos de dados. Este problema foi combatido com
    a combinação de algoritmos de aprendizagem não supervisionada que atuam
    como regularizadores durante a fase de pré treinamento.

  \item[saturação do gradiente:] Para poder aprender funções convexas
    complexas, é necessário que os neurônios da rede neural apliquem algum tipo
    de não linearidade em seu processamento. Porém, o algoritmo de
    \emph{backpropagation}, utilizado na aprendizagem, pode sofrer do fenômeno
    de saturação do gradiente. Isto ocorre em neurônios cujo valor de ativação
    estejam próximos dos extremos da função de ativação fazendo com que o
    gradiente fique próximo de zero. Esta saturação também propagada para as
    camadas mais inferiores da rede.

  \item[processamento intensivo:] Mesmo com estas técnicas, hoje em dia o
    processamento de redes profundas é predominantemente feito em GPUs, visto
    que estes hardwares são altamente especializados em processamento vetorial
    paralelo. O crescente suporte dos fabricantes destes hardwares e de
    frameworks e bibliotecas que facilitam a utilização dos mesmos também foi
    importante fator para tornar o uso de redes com múltiplas camadas viável.

\end{description}

\subsection{Aprendizado não supervisionado}

Aprendizado não supervisionado é usado para regularizar uma rede neural a fim
de impedir que a mesma apresente um alto grau de over fitting. Os algoritmos
mais empregados para esta tarefa são \emph{autoencoders} e \emph{Restricted
  Boltzman Machines}, que atuam no pré treinamento de redes neurais profundas
como regularizadores, fazendo uma pré calibragem dos parâmetros $W$ antes da
rede ser completamente conectada e \emph{fine tuned}. Este processo limita os
valores dos pesos das conexões da rede para um intervalo relativamente pequeno
e tem gerado bons resultados. Esta combinação de métodos supervisionados com
não supervisionados é conhecido como aprendizado semi supervisionado.

\subsubsection{Autoencoders}

Autoencoders são redes neurais de representação binária com uma única camada,
cuja \emph{loss function} usa a própria entrada como alvo de aprendizagem. Caso
o número de neurônios na camada intermediária seja menor que o número na camada
de entrada, então o autoencoder aprende uma representação mais compacta para os
dados de entrada, de fato, dependendo da função de erro usada, um autoencoder
realiza fatoração de componentes principais na entrada. Se o número de
neurônios na camada intermediária for maior ou igual ao número de neurônios na
entrada então os autoencoder não tem grande utilidade, exceto nos
\emph{denoising autoencoders} onde cada neurônio da entrada original é
corrompido (setado para 0) com uma probabilidade $p$. Isto confere ao
autoencoder um pouco de invariabilidade a ruído na rede, o que melhora a
generalização.

\begin{figure}
  \caption{Exemplo de Autoencoder}
  \begin{center}
    \includegraphics[scale=0.5]{placeholder}
  \end{center}
\end{figure}

\subsubsection{Restricted Boltzman Machines}

\emph{Restricted Boltzman Machine} é um exemplo de um formalismo conhecido como
Modelo Baseado em Energia (EBM). O objetivo de um EBM é encontrar uma função de
energia $E(X,Y)$ que associa um valor de entrada $X$ e um valor de saída $Y$ a
um escalar chamado de energia.  Uma boa função de energia é aquela que associa
um valor baixo para uma configuração $(X,Y)$ desejada, e um valor de energia
alto caso contrário.\citep{lecun2006tutorial}

Porém, uma Restricted Boltzman Machine é não supervisionada, portanto, a
variável $Y$ dos rótulos conhecidos é substituída por uma variável escondida
$h$. A representação mais comum de uma RBM é um grafo bipartido não direcionado
completamente conexo, onde uma das partições do grafo representa a entrada que
é visível $v$, e a outra representa a variáveis escondidas do modelo $h$.
Depois de um aprendizado realizado com sucesso, uma RBM contém a distribuição
probabilísticas das observações de treinamento.  Essa distribuição é expressa
pela probabilidade conjunta destas duas variáveis:

$$ p(v,h) = \frac{\exp(-E(v,h))}{Z}$$

Onde, $E(v,h)$ é a \emph{função de energia} que é dada pela fórmula:

$$E(v,h)=\mathbf{-bv^T-ch^T-h^{T}Wv}$$

Onde $\mathbf{b}$ e $\mathbf{c}$ são \emph{bias vectors} e $\mathbf{W}$ é a
matriz de pesos entre a camada visível e a camada escondida. Intuitivamente, o
objetivo destes parâmetros é armazenar uma preferência por uma configuração de
variáveis em particular. Além, disso a função de normalização $Z$ é dada por:

$$ Z = \sum_v \sum_h \exp(-E(v,h)) $$

Esta função geralmente não é tratável.

\begin{figure}
  \caption{Exemplo de RBM}
  \begin{center}
    \includegraphics[scale=0.5]{placeholder}
  \end{center}
\end{figure}

\subsection{Convolutional Neural Networks}

\subsubsection{Um breve histórico das CNNs}

Embora as CNNs já fossem conhecidas há bastante tempo, o modelo tradicional
para abordagem de problemas de classificação com \emph{Convolutional Neural
  Networks} envolvia o desenvolvimento de filtros criados de maneira
"artesanal"~por humanos. Na prática, isto limitava muito a aplicação de CNNs e
logo ficou clara a necessidade de um processo que pudesse realizar o
aprendizado destes filtros de maneira automática. A seguir, a listagem adaptada
de~\citet{cs231n}, mostra a evolução destas técnicas principalmente na última
década.

\begin{itemize}

  \item \emph{Neocognitron (1988)}: Criado por
    \citet{fukushima1988neocognitron}, foi o precursor das redes neurais
    convolucionais modernas. Consiste em múltiplas camadas de dois tipos de
    neurônios artificiais que tinham por objetivo simular a configuração
    observada por \citet{hubel1959receptive} em cérebros de gatos, os quais
    possuíam células responsáveis por tarefas distintas no reconhecimento
    visual.

  \item \emph{LeNet (1990)}: Proposta por \citet{le1990handwritten} foi usada
    com sucesso para reconhecimento de caracteres escritos a mão usando o
    dataset~MNIST.\@

  \item \emph{AlexNet (2012)}: \citet{krizhevsky2012imagenet} Foi a ganhadora
    da competição ILSVRC de 2012 e foi a responsável por grande parte do
    entusiasmo atual com redes convolucionais. Utiliza técnicas como
    \emph{droppout} e \emph{ReLUs}.

  \item \emph{ZF Net (2013)}: Se destacou na competição ILSVRC de 2013, por
    apresentar melhorias em relação à AlexNet, com a otimização de
    hyperparametros e a expansão do número de layers intermediários.
    \citet{zeiler2014visualizing}

  \item \emph{GoogLeNet (2014)}: A principal contribuição de
    \citet{szegedy2014going} foi o desenvolvimento de um \emph{Inception
      Module} que conseguiu reduzir o número de parâmetros (pesos) em 15 vezes.

  \item \emph{VGGNet (2014)}: \citet{zeiler2014visualizing} trouxe sua
    contribuição demonstrando que a profundidade da arquitetura é um componente
    crucial do desempenho da rede. Mais tarde foi descoberto que a sua
    performance supera a da GoogLeNet, embora necessite de mais memória e
    parâmetros. Hoje, é a rede preferida para extrair \emph{features} de
    imagens.

\end{itemize}

\subsubsection{VGGNet}

Até o presente momento, o modelo de \citep{zeiler2014visualizing} --- conhecido
como VGGNet --- é tido como um dos melhores, obtendo precisão semelhante à
humanos \citep{human}, com uma taxa de erro na tarefa de classificação da ILSVRC
de 6,8\% quando a classe correta da imagem não está entre as cinco primeiras
sugeridas pelo modelo. A competição deste ano (2015) ainda não teve seu
resultado anunciado, mas é provável que hajam novas melhorias nos modelos
atuais. Mesmo assim, a VGGNet ainda é um caso relevante para estudo.

\subsubsection{Arquitetura e Modelo}

A arquitetura usada pode ser representada da seguinte maneira:~\cite{cs231n}

\begin{footnotesize}
\begin{verbatim}
INPUT:     [224x224x3]   memory: 224*224*3=150K   weights: 0
CONV3-64:  [224x224x64]  memory: 224*224*64=3.2M  weights: (3*3*3)*64 = 1,728
CONV3-64:  [224x224x64]  memory: 224*224*64=3.2M  weights: (3*3*64)*64 = 36,864
POOL2:     [112x112x64]  memory: 112*112*64=800K  weights: 0
CONV3-128: [112x112x128] memory: 112*112*128=1.6M weights: (3*3*64)*128 = 73,728
CONV3-128: [112x112x128] memory: 112*112*128=1.6M weights: (3*3*128)*128 = 147,456
POOL2:     [56x56x128]   memory: 56*56*128=400K   weights: 0
CONV3-256: [56x56x256]   memory: 56*56*256=800K   weights: (3*3*128)*256 = 294,912
CONV3-256: [56x56x256]   memory: 56*56*256=800K   weights: (3*3*256)*256 = 589,824
CONV3-256: [56x56x256]   memory: 56*56*256=800K   weights: (3*3*256)*256 = 589,824
POOL2:     [28x28x256]   memory: 28*28*256=200K   weights: 0
CONV3-512: [28x28x512]   memory: 28*28*512=400K   weights: (3*3*256)*512 = 1,179,648
CONV3-512: [28x28x512]   memory: 28*28*512=400K   weights: (3*3*512)*512 = 2,359,296
CONV3-512: [28x28x512]   memory: 28*28*512=400K   weights: (3*3*512)*512 = 2,359,296
POOL2:     [14x14x512]   memory: 14*14*512=100K   weights: 0
CONV3-512: [14x14x512]   memory: 14*14*512=100K   weights: (3*3*512)*512 = 2,359,296
CONV3-512: [14x14x512]   memory: 14*14*512=100K   weights: (3*3*512)*512 = 2,359,296
CONV3-512: [14x14x512]   memory: 14*14*512=100K   weights: (3*3*512)*512 = 2,359,296
POOL2:     [7x7x512]     memory: 7*7*512=25K      weights: 0
FC:        [1x1x4096]    memory: 4096             weights: 7*7*512*4096 = 102,760,448
FC:        [1x1x4096]    memory: 4096             weights: 4096*4096 = 16,777,216
FC:        [1x1x1000]    memory: 1000             weights: 4096*1000 = 4,096,000

TOTAL memory: 24M * 4 bytes ~= 93MB / image (only forward! ~*2 for bwd)
TOTAL params: 138M parameters

\end{verbatim}
\end{footnotesize}

\emph{INPUT}, \emph{CONV}, \emph{POOL} e \emph{FC} representam camadas na rede
com funções específicas. INPUT, CONV e POOL são auto explicativos e FC
significa \emph{Fully Connected}. Os números ao lado do nome da camada se
referem a características dos filtros que devem ser aprendidos. CONV3\-64, por
exemplo, indica a presença de 64 filtros cuja distância espacial (altura e
largura) é igual a 3. Na camada de pooling, somente é representado a distância
espacial, ou seja 2.

As considerações de memória são cruciais visto que redes convolucionais
modernas dependem de placas gráficas para acelerar o seu treinamento e a
maioria destas possuem limites de memória que vão de 3 até 6GB\@.


O grande diferencial da VGGNet para a sua maior competidora, a GoogLeNet, é o
uso de filtros pequenos \emph{$F=3$}, os quais aceleram o treinamento da rede.
Além disso a rede apresenta uma profundidade (número de camadas) bastante
elevado (até 19), corroborando a ideia de que a profundidade da rede é
fundamental para um bom desempenho.

\subsection{Funcionamento}

Redes convolucionais são um tipo especial de \emph{deep networks} projetadas
especificamente para lidar com dados vetoriais tais como imagens e sons.  Este
tipo de rede neural ganhou fama na competição ILSVRC realizada em 2012 na qual
conseguiu-se diminuir pela metade o erro do melhor competidor. Este feito foi
atingido com o uso de ReLUs (\emph{rectified linear units}), GPUs e uma técnica
conhecida como \emph{dropout}. Desde então, Redes Convolucionais ganharam força
e hoje são as melhores técnicas para reconhecimento de
imagens.\cite{lecun2015deep}

Uma rede convolucional é baseada em 4 ideias principais: conexões locais, pesos
compartilhados, \emph{pooling} e múltiplas camadas. Conexões locais exploram o
fato de que em uma imagem existe uma correlação entre os valores dos pixels
vizinhos. Já pesos compartilhados, geram o conceito de \emph{feature maps} e
reduzem o número de pesos que precisam ser aprendidos e a operação de
\emph{pooling} (geralmente o máximo local) é usada pois é invariante a posição,
o que da um maior poder de abstração para a rede.\cite{lecun2015deep}

A operação de filtro é realizada pela função matemática de convolução, cujo
papel é detectar conjunções locais nas \emph{features} da camada anterior. Logo
em seguida, vem a camada de \emph{pooling} cujo papel é combinar
\emph{features} semanticamente semelhantes.\cite{lecun2015deep}

A rede é tipicamente composta de dois a três estágios de convolução seguidos da
aplicação de não linearidade e de \emph{pooling}. Por fim, são aplicadas mais
camadas convolucionais, desta vez, completamente conectadas. O aprendizado é
feito através de \emph{backpropagation} como em redes neurais
tradicionais.\cite{lecun2015deep}

\subsubsection{Hyper parâmetros}

O termo "hiper parâmetros" se refere as "alavancas" que um \emph{designer} pode
puxar enquanto decide a arquitetura da rede convolucional. Além dos hyper
parâmetros tradicionalmente conhecidos das redes neurais convencionais, tais
como, quantidade de neurônios por camada, função de não linearidade, taxa de
aprendizado e função de regularização, as redes convolucionais contém alguns
parâmetros que são exclusivos para o seu funcionamento:

\begin{itemize}

  \item \emph{Número de filtros ($K$):} Este parâmetro regula quantos filtros
    serão treinados na rede.
  \item \emph{Stride ($S$):} Se refere ao passo que é dado pela
    janela de convolução na imagem original.
  \item \emph{Distância espacial ($F$):}
    Se refere a largura e altura dos filtros.
  \item \emph{Padding ou Preenchimento
      ($P$):} Se refere a largura e altura dos filtros.

\end{itemize}

Porém nem toda combinação destes parâmetros é válida devido a propriedades
geométricas. Considerando um volume de entrada de dimensões: $W_1 \times H_1
\times D_1$, produzirá um volume de tamanho: $W_2 \times H_2 \times D_2$ sendo
que: $$ W_2= \frac{(W_1-F+2P)}{S+1} $$ $$ H_2= \frac{(H_1-F+2P)}{S+1} $$ $$
D_2= F $$ Uma combinação válida destes parâmetros produz um valor de $W_2$ (ou
$H_2$) inteiro.

Assumindo o uso de \emph{compartilhamento de parâmetros}, cada filtro adiciona
um total de $(F \cdot F \cdot D_1)\cdot K$ pesos e $K$ \emph{biases}.

A camada de pooling também possui seu próprio parâmetro $F$ e $S$ e seu
respectivo volume de saída terá as dimensões:

$$W_2 = (W_1 - F)/S + 1$$
$$H_2 = (H_1 - F)/S + 1$$
$$D_2 = D_1$$
\begin{figure}
  \caption{Exemplo de configuração espacial}
  \begin{center}
    \includegraphics[scale=0.5]{placeholder}
  \end{center}
\end{figure}
\subsubsection{Dropout}

O objetivo da técnica de \emph{dropout} é aproximar a solução ótima de calcular
a média dos resultados de treinamento de todas as combinações de parâmetros
possíveis. Isto é feito pela remoção aleatória e temporária de unidades da rede
bem como suas conexões. Cada unidade é mantida na rede com uma probabilidade
$p$, independente das outras.\cite{srivastava2014dropout}

Com a aplicação de \emph{dropout} há uma redução no \emph{overfitting} e uma
melhoria na regularização, porém, o treinamento da rede tende a demorar mais
gerando um \emph{tradeoff} entre \emph{overfitting} e tempo de
treinamento.\cite{srivastava2014dropout}

Está técnica não está restrita somente a redes \emph{feed forward} e pode ser
aplicadas a \emph{RBM} e \emph{autoencoders}, proporcionando uma maneira de
combinar exponencialmente diferentes arquiteturas
eficientemente.\cite{srivastava2014dropout}

\begin{figure}
  \caption{Uma rede neural com a aplicação de dropout}
  \begin{center}
    \includegraphics[scale=0.5]{placeholder}
  \end{center}
\end{figure}

\subsubsection{ReLU}

Em uma rede neural tradicional, as funções $\tanh(x)$ e $\frac{1}{1+e^{-x}}$
são usadas para introduzir não linearidade na computação do produto dos pesos
pelas entradas dos neurônios. Porém, estas funções possuem a desvantagem de
apresentarem o fenômeno de saturação. Isto ocorre quando a entrada da função
está distante da ponto central ($x=0$) e provoca uma perda de desempenho no
aprendizado.\cite{krizhevsky2012imagenet}

Por causa disto, o uso de \emph{rectifiers} tem se tornado cada vez mais comum
no treinamento de grandes redes. Este \emph{rectifier} é definido como:

$$ f(x) = \max(0,x) $$

Quando uma unidade ou neurônio usa este \emph{rectifier} ele é chamado de
Rectified Linear Unit ou ReLU.\@ De acordo com \citep{krizhevsky2012imagenet},
ReLUs são muito mais rápidos do que \emph{saturating non linearities} --- como
são conhecidas as funções mencionadas anteriormente --- e não requerem
regularização, embora isto ainda seja usado para aumentar o poder de
generalização da rede.

\begin{figure}
  \caption{Tipos de não linearidades aplicadas}
  \begin{center}
    \includegraphics[scale=0.5]{placeholder}
  \end{center}
\end{figure}

\subsubsection{Regularização}

Em uma típica rede \emph{deep}, existem milhares de pesos que sofrem adaptação
em seus valores. Normalmente um aproximador de função com este grande número de
parâmetros ajustáveis está sujeito a \emph{over fitting}. Isto reduz a
capacidade de generalização do modelo gerado para entradas que não fizeram
parte do treinamento. Redes neurais contornam este problema com o uso de
técnicas de regularização tais como:

\begin{itemize}

  \item \emph{Regularização L2}: Para cada peso $w$ da rede, é
    adicionado uma penalidade $\frac{1}{2}\lambda w^2$, onde $\lambda$ é a
    força de regularização. O termo $\frac{1}{2}$ é adicionado para facilitar o
    cálculo da derivada na fase de \emph{backpropagation} e o termo todo tem a
    interpretação de penalizar vetores de pesos com alguns poucos pesos muito
    altos e incentivar o uso de vetores difusos, ou seja, eles incentivam a
    rede a utilizar todos as suas entradas ao invés de usar demasiadamente uma
    entrada ou outra.

  \item \emph{Regularização L1}: Para cada peso $w$ da rede, é adicionado uma
    penalidade $\lambda|w|$ com a finalidade de tornar os pesos na rede
    esparsos (próximos de zero). Isto confere a rede um certo grau de
    invariabilidade à entradas com ruído. A regularização L1 pode ser usada
    juntamente com a L2, sendo chamadas neste caso de \emph{Elastic net
      regularization}.

  \item \emph{Max norm constraints}: Consiste em estipular um valor máximo para
    os pesos $w$ de maneira que a rede não se torne instável mesmo na presença
    de uma taxa de aprendizado muito grande.

  \item \emph{Dropout}: A técnica de \emph{dropout} também pode ser vista como
    uma regularização e pode ser usada juntamente com outros tipos.

\end{itemize}


\input{tex/demonstration-lenet}

\input{tex/conclusion}

%==============================================================================

\bibliographystyle{abntex2-alf}
\bibliography{biblio}

\end{document}

% chktex-file 18 disable warnings about "

\section{Deep Learning}

\subsection{Introdução}

A área de \emph{deep structured learning} que vem sendo conhecida como
\emph{deep learning} surgiu como um novo campo de pesquisa em aprendizado de
máquina nos últimos anos e as técnicas desenvolvidas nesta área já tem
impactado uma gama variada de trabalhos em processamento de sinais e
informações. Embora hajam diferentes definições, este trabalho adotará a
definição contida em \url{https://github.com/lisa-lab/DeepLearningTutorials}, visto
que o mesmo trabalho foi utilizado como base para este\cite{deng2014deep}:

\newtheorem{def-deep-learning}{Definição}

\begin{def-deep-learning}

  \begin{quote}\emph{Deep Learning} é uma nova área do aprendizado de máquina que vem sendo
introduzida com o objetivo de aproximar o Aprendizado de Máquina dos seu
objetivo original: Inteligência Artificial. \emph{Deep Learning} aborda
múltiplos níveis de representação e abstração que ajudam a processar dados tais
como sons, imagens e texto.
\end{quote}

\end{def-deep-learning}

% Origem do termo?

% Exemplos de aplicação

% Principais diferenças para as redes tradicionais.


\section{Why deep architetures}

% notes from lecun's https://www.youtube.com/watch?v=oOB4evKlEmQ

% Why deep learning instead of shallow learning since any deep architeture can
% be simulated by a shallow one?  According to yann le'cun it's because that is
% a trade-off between 'space' and computing time. It's true that shallow
% architetures can be simulated by a deeper one, but they run much much slower.

% See: Scaling learning algorithms towards AI

% Deep Supervised learning is hard beacuse the loss functions that are used in
% traing are non convex

% Backpropagation does not work well with networks that are tall and skinny
% lots of layers with few hidden units

% How to deal with limitations of linear classifiers You map the input to a
% higher dimension input where they are more easily separable

% How to build this: Use knowlege you already have to craft a feature set. We
% use a standard set of basis functions RBF Simplest approach use a kernel
% method

\chapter{Introdução}

Nos últimos anos houve um grande avanço na Computação Visual graças ao
ressurgimento de um campo da Inteligência Artificial conhecido como Redes
Neurais. Este progresso foi responsável pelo despertar do interesse e
investimentos de grandes empresas como Google, Facebook e outros que apostam no
desenvolvimento de aplicações que há poucas décadas atrás eram apenas tema de
ficção científica tais como scanners faciais de alta precisão, reconhecimento
de objetos em cenas, carros que não precisam de motorista e vários tipos de
classificadores em geral. Essas novas técnicas ficaram conhecidas como Deep
Learning e prometem fazer uso da enorme quantidade de dados e informações
disponíveis na internet e nas bases de dados das empresas.

Este progresso foi causado por fatores como o incremento da capacidade de
processamento dos computadores e a ubiquidade de tecnologias como GPUs que
facilitaram o processamento vetorial e paralelo de imagens, sons e outros
domínios de alta dimensionalidade que até então possuíam um custo computacional
proibitivo.

\section{Objetivos}

<REVISAR>
O objetivo deste trabalho consiste em efetuar uma revisão do
histórico do desenvolvimento das técnicas de Deep Learning e um resumo do seu
estado da arte com a finalidade de adquirir proficiência no uso das mesmas.
Além disso, serão apresentados resultados práticos obtidos com a execução de
uma modelo que utiliza Deep Learning chamado de Rede Neural Convolucional
(CNN) no reconhecimento de caracteres escritos à mão utilizando o algoritmo da
rede LeNet de LeCun et al sobre a base MNIST.\@

